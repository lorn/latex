\documentclass[a4paper,twoside]{article}      % Comments after  % are ignored
\usepackage{amsmath,amssymb,amsfonts} % Typical maths resource packages
\usepackage{graphics}                 % Packages to allow inclusion of graphics
\usepackage{color}                    % For creating coloured text and background
\usepackage{hyperref}                 % For creating hyperlinks in cross references

\oddsidemargin 0cm
\evensidemargin 0cm

\pagestyle{myheadings}         % Option to put page headers
                               % Needed \documentclass[a4paper,twoside]{article}
\markboth{{\small\it PDF\ \LaTeX \ coloured text and graphics}}
{{\small\it C.T.J. Dodson} }

\textwidth 15.5cm
\topmargin -1cm
\parindent 0cm
\textheight 24cm
\parskip 1mm
\newtheorem{theorem}{Theorem}[section]
\newtheorem{proposition}[theorem]{Proposition}
\newtheorem{corollary}[theorem]{Corollary}
\newtheorem{lemma}[theorem]{Lemma}
\newtheorem{remark}[theorem]{Remark}
\newtheorem{definition}[theorem]{Definition}

\def\R{\mathbb{ R}}
\def\S{\mathbb{ S}}

\date{\small\it May 16, 2000}
\title{Perl para Sysadmins e DBA''s}
%\footnote{A demonstration example including colored text and graphics}

\author{ Lindolfo ''Lorn'' Rodrigues \\
{\small Department of Mathematics, UMIST, Manchester M60 1QD, UK}
 }

\begin{document}
\maketitle
\begin{abstract}
{\color{red}
You are unlikely to want your research articles to use coloured text
but this illustrates how it can be done}
\end{abstract}

{\color{green}{\sc keywords: dpa, zero-knowledge, information theory,
distributions, metric}}

\section{Introduction}
Zero-Knowledge proofs allow verification of
secret-based actions without revealing the secrets. Goldreich et
al.~\cite{goldreich} discussed the class of promise problems in
which interaction may give additional information in the context
of Statistical Zero-Knowlege. They invoked two types of difference
between distributions: the `statistical difference' and the
`entropy difference' of two random variables. In this context,
typically, one of the distributions is the uniform distribution.

Thus, in the contexts of DPA and SZK tests, it is necessary to
compare two nearby distributions on bounded domains.
We describe the following result and discuss applications.
{\color{cyan}
\begin{proposition}\label{colour}
The family of probability density functions for random variable
$N\in [0,1]$ given by
\begin{equation}
g(N,\mu,\beta) = {\frac{{{{\frac{1}{N}}}^
       {1 - {\frac{\beta }{\mu }}}}\,
     ({{{\frac{\beta }{\mu }}})^{\beta }}\,
     {{(\log {\frac{1}{N}})}^
       { \beta -1}}}{
      \Gamma(\beta )}} \ \ \ {\rm for} \  \mu \  > 0 \ {\rm and} \ \beta \geq 1
  \label{loggamma}    \end{equation}
determines a metric space of distributions with the following
properties\\ $\bullet$ it contains the uniform distribution\\
$\bullet$ it contains approximations to truncated Gaussian
distributions\\ $\bullet$ the difference structure is given by the
information-theoretic metric\\ $\bullet$ as a Riemannian
2-manifold it is an isometric isomorph of the manifold of gamma
distributions.
\end{proposition} }

\section{In this example}
Examples are provided of
\begin{enumerate}
\item Coloured text~\pageref{colour}
\item Graphics~\pageref{graphics}
\item Tables~\pageref{table}
\end{enumerate}

\section{Proof of Proposition 1.1}
\subsection{Log-gamma PDFs}
By integration, it is easily checked that the family given by
equation (\ref{loggamma}) consists of probability density
functions for the random variable $N\in [0,1];$ some with central
mean are shown in Figure~\ref{pdf}. The limiting densities are
given by
\begin{eqnarray}
\lim_{\beta\rightarrow 1^+}g(N,\mu,\beta) =
g(N,\mu,1)&=&\frac{1}{\mu}\left(\frac{1}{N}\right)^
      {1 - \frac{1}{\mu }} \\
     \lim_{\mu\rightarrow 1}g(N,\mu,1) = g(N,1,1) &=& 1 \ .
\end{eqnarray}

\subsection{Information metric structure} For the log-gamma
densities, the Fisher information matrix determines a Riemannian
information metric~\cite{amari} on the parameter space ${\cal
S}=\{(\mu,\beta)\in (0,\infty)\times[1,\infty)\}.$ Its arc length
function is given by
\begin{equation}
ds_{\cal{S}}^2=\sum_{ij} g_{ij} \, dx^i dx^j =\frac{\beta}{\mu^2}
\, d\mu^2 +
        \left(\psi'(\beta) -\frac{1}{\beta}\right)\, d\beta^2 , \label{gammametric}
\end{equation}
where $\psi(\beta)=\frac{\Gamma'(\beta)}{\Gamma(\beta)}$ is the
logarithmic derivative of the gamma function, evaluated at
$\beta.$

In fact, (\ref{loggamma}) arises from the gamma family
\begin{equation}
f(x,\mu,\beta) = \frac{ x^{\beta -1} \,
     (\frac{\beta }{\mu})^\beta    }
     {\Gamma(\beta )} \,  e^{-\frac{V\,\beta }{\mu }}
     \label{gamma}
\end{equation}
for the non-negative random variable $x=\log\frac{1}{N}.$ It is
known that the gamma family (\ref{gamma}) has also the information
metric (\ref{gammametric}) (cf \cite{lauritzen}) so the identity
map on the space of coordinates $(\mu,\beta)$ is an isometry of
Riemannian manifolds. \hfill $\Box$

\section{Tables}\label{table}
Table~\ref{diffstruc} lists the number of differentiable
structures on spheres.
\begin{table}
\begin{center}
\framebox[1.5in]{\begin{tabular}{c || c | c }
$n$ & $\S^n$ & $\R^n$ \\ \hline
1 & 1 & 1 \\
2 & 1 & 1 \\
3 & 1 & 1 \\
4 & 1 & $\infty$ \\
5 & 1 & 1 \\
6 & 1 & 1 \\
7 & 28 & 1 \\
8 & 2 & 1 \\
9 & 8 & 1 \\
10 & 6 & 1 \\
11 & 992 & 1 \\
12 & 1 & 1 \\
13 & 3 & 1 \\
14 & 2 & 1 \\
15 & 16256 & 1 \\
\end{tabular}}
\caption{Numbers of distinct differentiable structures on real $n$-space
and $n$-spheres}
\label{diffstruc}
\end{center}
\end{table}

Here is how to set out a table in \LaTeX:
\begin{verbatim}
\begin{table}
\begin{center}
\framebox[1.5in]{\begin{tabular}{c | c | c }
$n$ & $\S^n$ & $\R^n$ \\ \hline
1 & 1 & 1 \\
2 & 1 & 1 \\
3 & 1 & 1 \\
4 & 1 & $\infty$ \\
5 & 1 & 1 \\
6 & 1 & 1 \\
7 & 28 & 1 \\
8 & 2 & 1 \\
9 & 8 & 1 \\
10 & 6 & 1 \\
11 & 992 & 1 \\
12 & 1 & 1 \\
13 & 3 & 1 \\
14 & 2 & 1 \\
15 & 16256 & 1 \\
\end{tabular}}
\caption{Numbers of distinct differentiable structures on real $n$-space
and $n$-spheres}
\label{diffstruc}
\end{center}
\end{table}
\end{verbatim}
This allowed us to cross-reference the Table via:
\begin{verbatim}
~\ref{diffstruc}
\end{verbatim}



\section{Graphics}\label{graphics}
\begin{figure}
\begin{picture}(300,220)(0,0)
%\put(-20,20){\resizebox{20 cm}{!}{\includegraphics{3dpdf}}}
%\put(260,30){\resizebox{15 cm}{!}{\includegraphics{contpdf}}}
\put(210,100){$\beta$}
\put(400,25){$N$}
\put(260,200){$\beta$}
\put(90,40){$N$}
\end{picture}
\caption{{\em The log-gamma family of densities with central mean
$<N> \, = \frac{1}{2}$ as a surface and as a contour plot. }}
\label{pdf}
\end{figure}


Figure~\ref{pdf} was created by calling in the two pdf graphics files
{\tt 3dpdf.pdf} , {\tt contpdf.pdf}
placed together in the following {\tt picture} environment
inside a {\tt figure} environment with a caption and label:
\begin{verbatim}
\begin{figure}
\begin{picture}(300,220)(0,0)
\put(210,100){$\beta$}
\put(400,25){$N$}
\put(260,200){$\beta$}
\put(90,40){$N$}
\end{picture}
\caption{{\em The log-gamma family of densities with central mean
$<N> \, = \frac{1}{2}$ as a surface and as a contour plot. }}
\label{pdf}
\end{figure}
\end{verbatim}






\begin{thebibliography}{99}
\bibitem{amari} S-I. Amari. {\bf Differential Geometrical Methods in
Statistics}, Springer Lecture Notes in Statistics 28,
Springer-Verlag, Berlin 1985.

\bibitem{chari} S. Chari, C.S. Jutla, J.R. Rao and P. Rohatgi.
Towards sound approaches to counteract power-analysis attacks. In
{\bf Advances in Cryptology-CRYPTO '99}, Ed. M. Wiener, Lecture
Notes in Computer Science 1666, Springer, Berlin 1999 pp 398-412.

\bibitem{crescenzo} G. Di Crescenzo and R. Ostrovsky.
On concurrent zero-knowledge with pre-processing. In {\bf Advances
in Cryptology-CRYPTO '99}  Ed. M. Wiener, Lecture Notes in
Computer Science 1666, Springer, Berlin 1999 pp 485-502.

\bibitem{tg} C.T.J. Dodson and T. Poston. {\bf Tensor Geometry}, Graduate
Texts in Mathematics 130, Second edition, Springer-Verlag, New
York, 1991. \htmladdnormallink{http://www.ma.umist.ac.uk/kd/tg.html}
{http://www.ma.umist.ac.uk/kd/tg.html}


\bibitem{goldreich} O. Goldreich, A. Sahai and S. Vadham.
Can Statistical Zero-Knowledge be made non-interactive? Or, on the
relationship of SZK and NISZK. In {\bf Advances in
Cryptology-CRYPTO '99},  Ed. M. Wiener, Lecture Notes in Computer
Science 1666, Springer, Berlin 1999 pp 467-484.

\bibitem{kocher} P. Kocher, J. Jaffe and B.Jun.
Differential Power Analysis. In {\bf Advances in Cryptology-CRYPTO
'99},  Ed. M. Wiener, Lecture Notes in Computer Science 1666,
Springer, Berlin 1999 pp 388-397.

\bibitem{lauritzen} S.L. Lauritzen. Statistical Manifolds. In {\bf
Differential Geometry in Statistical Inference}, Institute of
Mathematical Statistics Lecture Notes, Volume 10, Berkeley 1987,
pp 163-218.

\end{thebibliography}
\end{document}
