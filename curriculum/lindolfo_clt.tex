%%%%%%%%%%%%%%%%%%%%%%%%%%%%%%%%%%%%%%%%%%%%%%%%%%%%%%%%%%%%%%%%%%%%%%%%
%%%%%%%%%%%%%%%%%%%%%% Simple LaTeX CV Template %%%%%%%%%%%%%%%%%%%%%%%%
%%%%%%%%%%%%%%%%%%%%%%%%%%%%%%%%%%%%%%%%%%%%%%%%%%%%%%%%%%%%%%%%%%%%%%%%

%%%%%%%%%%%%%%%%%%%%%%%%%%%%%%%%%%%%%%%%%%%%%%%%%%%%%%%%%%%%%%%%%%%%%%%%
%% NOTE: If you find that it says                                     %%
%%                                                                    %%
%%                           1 of ??                                  %%
%%                                                                    %%
%% at the bottom of your first page, this means that the AUX file     %%
%% was not available when you ran LaTeX on this source. Simply RERUN  %% 
%% LaTeX to get the ``??'' replaced with the number of the last page  %% 
%% of the document. The AUX file will be generated on the first run   %%
%% of LaTeX and used on the second run to fill in all of the          %%
%% references.                                                        %%
%%%%%%%%%%%%%%%%%%%%%%%%%%%%%%%%%%%%%%%%%%%%%%%%%%%%%%%%%%%%%%%%%%%%%%%%

%%%%%%%%%%%%%%%%%%%%%%%%%%%% Document Setup %%%%%%%%%%%%%%%%%%%%%%%%%%%%

% Don't like 10pt? Try 11pt or 12pt
\documentclass[10pt]{article}

% This is a helpful package that puts math inside length specifications
\usepackage{calc}

% Layout: Puts the section titles on left side of page
\reversemarginpar

%
%         PAPER SIZE, PAGE NUMBER, AND DOCUMENT LAYOUT NOTES:
%
% The next \usepackage line changes the layout for CV style section
% headings as marginal notes. It also sets up the paper size as either
% letter or A4. By default, letter was used. If A4 paper is desired,
% comment out the letterpaper lines and uncomment the a4paper lines.
%
% As you can see, the margin widths and section title widths can be
% easily adjusted.
%
% ALSO: Notice that the includefoot option can be commented OUT in order
% to put the PAGE NUMBER *IN* the bottom margin. This will make the
% effective text area larger.
%
% IF YOU WISH TO REMOVE THE ``of LASTPAGE'' next to each page number,
% see the note about the +LP and -LP lines below. Comment out the +LP
% and uncomment the -LP.
%
% IF YOU WISH TO REMOVE PAGE NUMBERS, be sure that the includefoot line
% is uncommented and ALSO uncomment the \pagestyle{empty} a few lines
% below.
%

%% Use these lines for letter-sized paper
\usepackage[paper=letterpaper,
            %includefoot, % Uncomment to put page number above margin
            marginparwidth=1.2in,     % Length of section titles
            marginparsep=.05in,       % Space between titles and text
            margin=1in,               % 1 inch margins
            includemp]{geometry}
\usepackage[utf8]{inputenc}
\usepackage[brazil]{babel}

%% Use these lines for A4-sized paper
%\usepackage[paper=a4paper,
%            %includefoot, % Uncomment to put page number above margin
%            marginparwidth=30.5mm,    % Length of section titles
%            marginparsep=1.5mm,       % Space between titles and text
%            margin=25mm,              % 25mm margins
%            includemp]{geometry}

%% More layout: Get rid of indenting throughout entire document
\setlength{\parindent}{0in}

%% This gives us fun enumeration environments. compactitem will be nice.
\usepackage{paralist}

%% Reference the last page in the page number
%
% NOTE: comment the +LP line and uncomment the -LP line to have page
%       numbers without the ``of ##'' last page reference)
%
% NOTE: uncomment the \pagestyle{empty} line to get rid of all page
%       numbers (make sure includefoot is commented out above)
%
\usepackage{fancyhdr,lastpage}
\pagestyle{fancy}
%\pagestyle{empty}      % Uncomment this to get rid of page numbers
\fancyhf{}\renewcommand{\headrulewidth}{0pt}
\fancyfootoffset{\marginparsep+\marginparwidth}
\newlength{\footpageshift}
\setlength{\footpageshift}
          {0.5\textwidth+0.5\marginparsep+0.5\marginparwidth-2in}
\lfoot{\hspace{\footpageshift}%
       \parbox{4in}{\, \hfill %
                    \arabic{page} of \protect\pageref*{LastPage} % +LP
%                    \arabic{page}                               % -LP
                    \hfill \,}}

% Finally, give us PDF bookmarks
\usepackage{color,hyperref}
\definecolor{darkblue}{rgb}{0.0,0.0,0.3}
\hypersetup{colorlinks,breaklinks,
            linkcolor=darkblue,urlcolor=darkblue,
            anchorcolor=darkblue,citecolor=darkblue}

%%%%%%%%%%%%%%%%%%%%%%%% End Document Setup %%%%%%%%%%%%%%%%%%%%%%%%%%%%


%%%%%%%%%%%%%%%%%%%%%%%%%%% Helper Commands %%%%%%%%%%%%%%%%%%%%%%%%%%%%

% The title (name) with a horizontal rule under it
%
% Usage: \makeheading{name}
%
% Place at top of document. It should be the first thing.
\newcommand{\makeheading}[1]%
        {\hspace*{-\marginparsep minus \marginparwidth}%
         \begin{minipage}[t]{\textwidth+\marginparwidth+\marginparsep}%
                {\large \bfseries #1}\\[-0.15\baselineskip]%
                 \rule{\columnwidth}{1pt}%
         \end{minipage}}

% The section headings
%
% Usage: \section{section name}
%
% Follow this section IMMEDIATELY with the first line of the section
% text. Do not put whitespace in between. That is, do this:
%
%       \section{My Information}
%       Here is my information.
%
% and NOT this:
%
%       \section{My Information}
%
%       Here is my information.
%
% Otherwise the top of the section header will not line up with the top
% of the section. Of course, using a single comment character (%) on
% empty lines allows for the function of the first example with the
% readability of the second example.
\renewcommand{\section}[2]%
        {\pagebreak[2]\vspace{1.3\baselineskip}%
         \phantomsection\addcontentsline{toc}{section}{#1}%
         \hspace{0in}%
         \marginpar{
         \raggedright \scshape #1}#2}

% An itemize-style list with lots of space between items
\newenvironment{outerlist}[1][\enskip\textbullet]%
        {\begin{itemize}[#1]}{\end{itemize}%
         \vspace{-.6\baselineskip}}

% An environment IDENTICAL to outerlist that has better pre-list spacing
% when used as the first thing in a \section 
\newenvironment{lonelist}[1][\enskip\textbullet]%
        {\vspace{-\baselineskip}\begin{list}{#1}{%
        \setlength{\partopsep}{0pt}%
        \setlength{\topsep}{0pt}}}
        {\end{list}\vspace{-.6\baselineskip}}

% An itemize-style list with little space between items
\newenvironment{innerlist}[1][\enskip\textbullet]%
        {\begin{compactitem}[#1]}{\end{compactitem}}

% To add some paragraph space between lines.
% This also tells LaTeX to preferably break a page on one of these gaps
% if there is a needed pagebreak nearby.
\newcommand{\blankline}{\quad\pagebreak[2]}

%%%%%%%%%%%%%%%%%%%%%%%% End Helper Commands %%%%%%%%%%%%%%%%%%%%%%%%%%%

%%%%%%%%%%%%%%%%%%%%%%%%% Begin CV Document %%%%%%%%%%%%%%%%%%%%%%%%%%%%

\begin{document}
\makeheading{Lindolfo Rodrigues de Oliveira Neto}

\section{Contato}
%
% NOTE: Mind where the & separators and \\ breaks are in the following
%       table.
%
% ALSO: \rcollength is the width of the right column of the table 
%       (adjust it to your liking; default is 1.85in).
%
\newlength{\rcollength}\setlength{\rcollength}{1.85in}%
%
\begin{tabular}[t]{@{}p{\textwidth-\rcollength}p{\rcollength}}
%\href{http://www.ece.osu.edu/}%
%     {Department of Electrical and Computer Engineering} & \\
%\href{http://www.osu.edu/}{The Ohio State University}
Casado, 23 anos   & \textit{Telefone:} (+5511) 8147-6626 \\
Residência: Bairro Tucuruvi, próximo ao metrô & \textit{E-mail:}
\href{mailto:lorn@perl.org.br}{lorn@perl.org.br}\\
   & \textit{WWW:}
\href{http://lornlab.org/}{www.lornlab.org}\\
\end{tabular}

\section{Objetivo} 

\begin{innerlist}
\item Desenvolvimento de Software, Arquitetura de Sistemas e Sistemas
distribuídos com alta perfomance e acesso
\end{innerlist}

\section{Educação}
%
\href{http://fasp.br/}{\textbf{FASP}}, 
São Paulo, São Paulo Brazil
\begin{outerlist}

\item[] B., 
        \href{http://fasp.br/}
             {Ciência da Computação}, Junho 2007

\end{outerlist}


\section{Domínio}

\begin{innerlist}
        \item Desenvolvimento
        \begin{innerlist}
            \item Perl Moderno
                \begin{innerlist}
                    \item Moose
                    \item Catalyst - Framework MVC 
                    \item DBI/DBIx Interfaceando com - PostgreSQL ,MySQL e Oracle 
                \end{innerlist}
        \end{innerlist}
%\item XS - Otimizações de programas Perl usando a linguagem C
        \item Infra Estrutura
            \begin{innerlist}
                \item Apache 2x ( Compilações otimizadas para o ambiente, mod\_perl, mod\_php )
                \item Firewall( Iptables, pf, ipfw ) 
                \item DNS com Bind
                \item Servidor de Proxy interno e externo ( Proxy reverso )
                \item Servidor de Email ( Postfix )
                \item Monitoramento ( Nagios, Cacti, MRTG )
            \end{innerlist}
\end{innerlist}

\section{Bons Conhecimentos}

\begin{innerlist}
\item MySQL ( Criação de ''user functions'' - Levenstein - para NLP )
\item PostgreSQL ( Uso de PL/Perl para rodar rotinas pesadas )
\item Solaris, *BSD, Linux (Slackware, Red Hat, Debian)
\item PHP, Java, Linguagem C e integração Perl/C ( Perl XS )

\end{innerlist}

\section{Conhecimento}

\begin{innerlist}
\item Python, Smalltalk, Haskell ( Alguns estudo e uso/testes em projetos
particulares )

\end{innerlist}
%

\blankline

\section{Atividades extra curriculares}

\begin{innerlist}

\item Aula de Recuperação de Informação no IME-USP 01/2008 a 06/2008
\item \href{http://search.cpan.org/~lorn/}{\textbf{Módulos no CPAN - http://search.cpan.org/~lorn/}} 
\item Organização do YAPC - Yet Another Perl Conference 2006, 2007 e 2008
\item Organização do Slackshow - Encontro de usuários Slackware 2005, 2006, 2007 e 2008
\end{innerlist}

\section{Linguas}
\begin{innerlist}
\item Inglês para Leitura, Escrita e Conversação
\end{innerlist}

\section{Palestras}
\begin{innerlist}
\item Perl para Sysadmins e DBAs - FISL 10, Porto Alegre (Junho/2009)
        \begin{innerlist}
        \item[] Como usar o Perl para automatizar as tarefas diárias e
        fatídicas de um Sysadmin e como um DBA pode usar o poder NLP de Perl para arrumar
        os dados de seu banco de dados
        \end{innerlist}
\item Catalyst Framework - FISL 8, Porto Alegre (Abril/2007)
        \begin{innerlist}
        \item[] Apresentação sobre as funções do Catalyst
				 e sua flexibilidade com o MVC.
        \end{innerlist}
\item Backup com Bacula - Slackshow 2007 (Agosto/2007)
        \begin{innerlist}
        \item[] Como usar o Bacula, software de backup livre, para gerenciar
	 o backup de todos os servidores e computadores de uma empresa.
        \end{innerlist}

\item Webservices de spiders com Catalyst - 
	Encontro Perl na Faculdade Impacta (Dezembro/2007)
        \begin{innerlist}
        \item[] Usando Webservices em REST para criar servicos baseados em
        spiders.
        \end{innerlist}
\blankline
\blankline
\item  LWP::Curl - YAPC::Brasil::2008 (Outubro/2008)
        \begin{innerlist}
        \item[] Novo engine de spider baseado no Curl,
			 de 3 a 4 vezes mais rapido que o LWP/Mechanize.
        \end{innerlist}
\end{innerlist}

\section{Experiência Profissional}

\href{http://www.jacotei.com.br/}{\textbf{JáCotei}}, 
São Paulo, São Paulo Brasil

\begin{outerlist}
\item[] \textit{Gerente de TI}%
        \hfill \textbf{04/2009 até presente data}
\begin{innerlist}
\item Gerenciamento da equipe de desenvolvimento e infraestrutura
\item Migração do site de um framework antigo ( XmlNuke ) para um novo, MVC e
moderno ( CakePHP )
\item Re-estruturação da infraestrutura para melhorar a contingência e a
disponibilidade do site, com Load Balancer, uso de máquinas virtuais, etc
\end{innerlist}
\end{outerlist}

\begin{outerlist}
\item[] \textit{Desenvolvedor}%
        \hfill \textbf{01/2009 até 03/2009}
\begin{innerlist}
\item Gerencia dos projetos ( Infra/Dev )  com Scrum
\item Manutenção dos spiders das lojas
\item Projeto de consolidação de 2 ambientes de spiders ( um escrito em 2000 e
outro em 2005 )
\end{innerlist}
\end{outerlist}



\href{http://www.uplexis.com/}{\textbf{Uplexis}}, 
São Paulo, São Paulo Brasil
\begin{outerlist}
\item[] \textit{Desenvolvedor}%
        \hfill \textbf{06/2007 at 01/2009}
\begin{innerlist}
\item Scrum, TDD
\item Criação de manutenção de spiders de alta performance, transformando em
serviço acessado via Webservice REST
\item Uso de Inteligência Artificial para quebrar captchas
\item Manutenção da infraestrutura de 15 servidores ( Firewall, Apache,
Postfix )
\item Criação de um sistema de busca para o Diario Oficial de São Paulo ( DOSP
) usando Lucene como base.
\item Uso de Perl com NLP para limpeza de banco de dados
\end{innerlist}
\end{outerlist}

\href{http://www.clipdo.com.br/}{\textbf{ClipDO}}, 
São Paulo, São Paulo Brasil
\begin{outerlist}
\item[] \textit{Desenvolvedor}%
        \hfill \textbf{01/2006 at 06/2007}
\begin{innerlist}
\item Criação de sistema customizado para o Bradesco
\item Manutenção e re-estruturação dos programas internos escritos em Perl
\item Manutenção da infraestrutura de 15 servidores ( Firewall, Apache,
Postfix )
\end{innerlist}
\end{outerlist}
%  A empresa tem um mecanismo de pesquisa simultanea em sites de companias aéreas para comparação de preco de passagem.



\end{document}

%%%%%%%%%%%%%%%%%%%%%%%%%% End CV Document %%%%%%%%%%%%%%%%%%%%%%%%%%%%%
