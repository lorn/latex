%%%%%%%%%%%%%%%%%%%%%%%%%%%%%%%%%%%%%%%%%%%%%%%%%%%%%%%%%%%%%%%%%%%%%%%%
%%%%%%%%%%%%%%%%%%%%%% Simple LaTeX CV Template %%%%%%%%%%%%%%%%%%%%%%%%
%%%%%%%%%%%%%%%%%%%%%%%%%%%%%%%%%%%%%%%%%%%%%%%%%%%%%%%%%%%%%%%%%%%%%%%%

%%%%%%%%%%%%%%%%%%%%%%%%%%%%%%%%%%%%%%%%%%%%%%%%%%%%%%%%%%%%%%%%%%%%%%%%
%% NOTE: If you find that it says                                     %%
%%                                                                    %%
%%                           1 of ??                                  %%
%%                                                                    %%
%% at the bottom of your first page, this means that the AUX file     %%
%% was not available when you ran LaTeX on this source. Simply RERUN  %% 
%% LaTeX to get the ``??'' replaced with the number of the last page  %% 
%% of the document. The AUX file will be generated on the first run   %%
%% of LaTeX and used on the second run to fill in all of the          %%
%% references.                                                        %%
%%%%%%%%%%%%%%%%%%%%%%%%%%%%%%%%%%%%%%%%%%%%%%%%%%%%%%%%%%%%%%%%%%%%%%%%

%%%%%%%%%%%%%%%%%%%%%%%%%%%% Document Setup %%%%%%%%%%%%%%%%%%%%%%%%%%%%

% Don't like 10pt? Try 11pt or 12pt
\documentclass[10pt]{article}

% This is a helpful package that puts math inside length specifications
\usepackage{calc}

% Layout: Puts the section titles on left side of page
\reversemarginpar

%
%         PAPER SIZE, PAGE NUMBER, AND DOCUMENT LAYOUT NOTES:
%
% The next \usepackage line changes the layout for CV style section
% headings as marginal notes. It also sets up the paper size as either
% letter or A4. By default, letter was used. If A4 paper is desired,
% comment out the letterpaper lines and uncomment the a4paper lines.
%
% As you can see, the margin widths and section title widths can be
% easily adjusted.
%
% ALSO: Notice that the includefoot option can be commented OUT in order
% to put the PAGE NUMBER *IN* the bottom margin. This will make the
% effective text area larger.
%
% IF YOU WISH TO REMOVE THE ``of LASTPAGE'' next to each page number,
% see the note about the +LP and -LP lines below. Comment out the +LP
% and uncomment the -LP.
%
% IF YOU WISH TO REMOVE PAGE NUMBERS, be sure that the includefoot line
% is uncommented and ALSO uncomment the \pagestyle{empty} a few lines
% below.
%

%% Use these lines for letter-sized paper
\usepackage[paper=letterpaper,
            %includefoot, % Uncomment to put page number above margin
            marginparwidth=1.2in,     % Length of section titles
            marginparsep=.05in,       % Space between titles and text
            margin=1in,               % 1 inch margins
            includemp]{geometry}
%\usepackage[utf8]{inputenc}
\usepackage[brazil]{babel}

%% Use these lines for A4-sized paper
%\usepackage[paper=a4paper,
%            %includefoot, % Uncomment to put page number above margin
%            marginparwidth=30.5mm,    % Length of section titles
%            marginparsep=1.5mm,       % Space between titles and text
%            margin=25mm,              % 25mm margins
%            includemp]{geometry}

%% More layout: Get rid of indenting throughout entire document
\setlength{\parindent}{0in}

%% This gives us fun enumeration environments. compactitem will be nice.
\usepackage{paralist}

%% Reference the last page in the page number
%
% NOTE: comment the +LP line and uncomment the -LP line to have page
%       numbers without the ``of ##'' last page reference)
%
% NOTE: uncomment the \pagestyle{empty} line to get rid of all page
%       numbers (make sure includefoot is commented out above)
%
\usepackage{fancyhdr,lastpage}
\pagestyle{fancy}
%\pagestyle{empty}      % Uncomment this to get rid of page numbers
\fancyhf{}\renewcommand{\headrulewidth}{0pt}
\fancyfootoffset{\marginparsep+\marginparwidth}
\newlength{\footpageshift}
\setlength{\footpageshift}
          {0.5\textwidth+0.5\marginparsep+0.5\marginparwidth-2in}
\lfoot{\hspace{\footpageshift}%
       \parbox{4in}{\, \hfill %
                    \arabic{page} of \protect\pageref*{LastPage} % +LP
%                    \arabic{page}                               % -LP
                    \hfill \,}}

% Finally, give us PDF bookmarks
\usepackage{color,hyperref}
\definecolor{darkblue}{rgb}{0.0,0.0,0.3}
\hypersetup{colorlinks,breaklinks,
            linkcolor=darkblue,urlcolor=darkblue,
            anchorcolor=darkblue,citecolor=darkblue}

%%%%%%%%%%%%%%%%%%%%%%%% End Document Setup %%%%%%%%%%%%%%%%%%%%%%%%%%%%


%%%%%%%%%%%%%%%%%%%%%%%%%%% Helper Commands %%%%%%%%%%%%%%%%%%%%%%%%%%%%

% The title (name) with a horizontal rule under it
%
% Usage: \makeheading{name}
%
% Place at top of document. It should be the first thing.
\newcommand{\makeheading}[1]%
        {\hspace*{-\marginparsep minus \marginparwidth}%
         \begin{minipage}[t]{\textwidth+\marginparwidth+\marginparsep}%
                {\large \bfseries #1}\\[-0.15\baselineskip]%
                 \rule{\columnwidth}{1pt}%
         \end{minipage}}

% The section headings
%
% Usage: \section{section name}
%
% Follow this section IMMEDIATELY with the first line of the section
% text. Do not put whitespace in between. That is, do this:
%
%       \section{My Information}
%       Here is my information.
%
% and NOT this:
%
%       \section{My Information}
%
%       Here is my information.
%
% Otherwise the top of the section header will not line up with the top
% of the section. Of course, using a single comment character (%) on
% empty lines allows for the function of the first example with the
% readability of the second example.
\renewcommand{\section}[2]%
        {\pagebreak[2]\vspace{1.3\baselineskip}%
         \phantomsection\addcontentsline{toc}{section}{#1}%
         \hspace{0in}%
         \marginpar{
         \raggedright \scshape #1}#2}

% An itemize-style list with lots of space between items
\newenvironment{outerlist}[1][\enskip\textbullet]%
        {\begin{itemize}[#1]}{\end{itemize}%
         \vspace{-.6\baselineskip}}

% An environment IDENTICAL to outerlist that has better pre-list spacing
% when used as the first thing in a \section 
\newenvironment{lonelist}[1][\enskip\textbullet]%
        {\vspace{-\baselineskip}\begin{list}{#1}{%
        \setlength{\partopsep}{0pt}%
        \setlength{\topsep}{0pt}}}
        {\end{list}\vspace{-.6\baselineskip}}

% An itemize-style list with little space between items
\newenvironment{innerlist}[1][\enskip\textbullet]%
        {\begin{compactitem}[#1]}{\end{compactitem}}

% To add some paragraph space between lines.
% This also tells LaTeX to preferably break a page on one of these gaps
% if there is a needed pagebreak nearby.
\newcommand{\blankline}{\quad\pagebreak[2]}

%%%%%%%%%%%%%%%%%%%%%%%% End Helper Commands %%%%%%%%%%%%%%%%%%%%%%%%%%%

%%%%%%%%%%%%%%%%%%%%%%%%% Begin CV Document %%%%%%%%%%%%%%%%%%%%%%%%%%%%

\begin{document}
\makeheading{Lindolfo Rodrigues de Oliveira Neto}

\section{Contato}
%
% NOTE: Mind where the & separators and \\ breaks are in the following
%       table.
%
% ALSO: \rcollength is the width of the right column of the table 
%       (adjust it to your liking; default is 1.85in).
%
\newlength{\rcollength}\setlength{\rcollength}{1.85in}%
%
\begin{tabular}[t]{@{}p{\textwidth-\rcollength}p{\rcollength}}
%\href{http://www.ece.osu.edu/}%
%     {Department of Electrical and Computer Engineering} & \\
%\href{http://www.osu.edu/}{The Ohio State University}
Casado, 23 anos   & \textit{Telefone:} (+5511) 8147-6626 \\
Residencia: Bairro Tucuruvi, proximo ao metro & \textit{E-mail:}
\href{mailto:lorn@perl.org.br}{lorn@perl.org.br}\\
   & \textit{WWW:}
\href{http://lornlab.org/}{www.lornlab.org}\\
\end{tabular}

\section{Objetivo}
%
Perl Developer

\section{Educacion}
%
\href{http://fasp.br/}{\textbf{FASP}}, 
Sao Paulo, Sao Paulo Brazil
\begin{outerlist}

\item[] B.S, 
        \href{http://fasp.br/}
             {Computer Science}, Junho 2007

\end{outerlist}

\section{Conhecimentos} 

\begin{innerlist}
\item 4 anos de experiencia em desenvolvimento.
\item Solidos conhecimento em Banco de dados ( Oracle, PostgreSQL, MySQL )
\end{innerlist}

\section{Dominio}

\begin{innerlist}
        \item Desenvolvimento
        \begin{innerlist}
            \item Perl Moderno
                \begin{innerlist}
                    \item Moose
                    \item Catalyst - Framework MVC 
                    \item DBI/DBIx Interfaceando com - PostgreSQL ,MySQL e Oracle 
                \end{innerlist}
%\item XS - Otimiza��es de programas Perl usando a linguagem C
            \item Lucene Java 
                \begin{innerlist}
                    \item lorem ipsum
                    \item Catalyst - Framework MVC 
                \end{innerlist}
        \end{innerlist}
        \item Infra Estrutura
            \begin{innerlist}
                \item Apache 2x ( Compila��es otimizadas para o ambiente, mod\_perl, mod\_php )
                \item Firewall( Iptables, pf, ipfw ) 
                \item DNS com Bind
                \item Servidor de Proxy interno e externo ( Proxy reverso )
                \item Servidor de Email ( Postfix )
                \item Monitoramento ( Nagios, Cacti, MRTG )
            \end{innerlist}
\end{innerlist}

\section{Bons Conhecimentos}

\begin{innerlist}
\item MySQL ( Cria��o de ''user functions'' - Levenstein - para NLP )
\item PostgreSQL ( Uso de PL/Perl para rodar rotinas pesadas )
\item Solaris, *BSD, Linux (Slackware, Red Hat, Debian)
\item PHP, Java

\end{innerlist}

\section{Conhecimento}

\begin{innerlist}
\item Python, Smalltalk, Haskell ( Alguns estudo e
\item MySQL (versões 4.x, 5.0, 5.1 )
\item Solaris, *BSD, Linux (Slackware, Red Hat, Debian)
\item PHP, Java

\end{innerlist}
%

\blankline

\section{Atividades extra curriculares}

\begin{innerlist}

\item \href{http://search.cpan.org/~lorn/}{\textbf{Módulos no CPAN - http://search.cpan.org/~lorn/}} 
\item Organização do YAPC - Yet Another Perl Conference 2006, 2007 e 2008
\item Organização do Slackshow - Encontro de usuários Slackware 2005, 2006, 2007 e 2008
\end{innerlist}

\section{Linguas}
\begin{innerlist}
\item Inglês para Leitura, Escrita e Conversação
\end{innerlist}

\section{Palestras}
\begin{innerlist}
\item Perl para Sysadmins e DBAs - FISL 10, Porto Alegre (Junho/2009)
        \begin{innerlist}
        \item[] Apresentação sobre as funções do Catalyst
				 e sua flexibilidade com o MVC.
        \end{innerlist}
\item Catalyst Framework - FISL 8, Porto Alegre (Abril/2007)
        \begin{innerlist}
        \item[] Apresentação sobre as funções do Catalyst
				 e sua flexibilidade com o MVC.
        \end{innerlist}
\item Backup com Bacula - Slackshow 2007 (Agosto/2007)
        \begin{innerlist}
        \item[] Como usar o Bacula, software de backup livre, para gerenciar
	 o backup de todos os servidores e computadores de uma empresa.
        \end{innerlist}

\item Webservices de spiders com Catalyst - 
	Encontro Perl na Faculdade Impacta (Dezembro/2007)
        \begin{innerlist}
        \item[] Usando Webservices em REST para criar servicos baseados em spider.
        \end{innerlist}
\blankline
\blankline
\item  LWP::Curl - YAPC::Brasil::2008 (Outubro/2008)
        \begin{innerlist}
        \item[] Novo engine de spider baseado no Curl,
			 de 3 a 4 vezes mais rapido que o LWP/Mechanize.
        \end{innerlist}
\end{innerlist}

\section{Experi�ncia Profissional}

\href{http://www.voofacil.com/}{\textbf{VôoFácil}}, 
S�o Paulo, S�o Paulo Brasil
\begin{outerlist}
\item[] \textit{Desenvolvedor Perl}%
	\hfill \textbf{PJ 20h fixas/mês}
        \hfill \textbf{03/2007 até 01/2008}
\begin{innerlist}
\item Manutenção nos spiders das companias aéreas
\item Novos spiders para novas companias
\item Integração dos spiders em Perl com o site que é em PHP
\end{innerlist}

\end{outerlist}
%  A empresa tem um mecanismo de pesquisa simultanea em sites de companias aéreas para comparação de preco de passagem.



\end{document}

%%%%%%%%%%%%%%%%%%%%%%%%%% End CV Document %%%%%%%%%%%%%%%%%%%%%%%%%%%%%
